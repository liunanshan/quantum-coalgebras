\documentclass[a4paper, 11pt]{article}
\usepackage{comment} % enables the use of multi-line comments (\ifx \fi) 
\usepackage{lipsum} %This package just generates Lorem Ipsum filler text. 
\usepackage{fullpage} % changes the margin
\usepackage{libertine}
\usepackage{color}
\newcommand{\nevComment}[1]{\textcolor{red}{RN: #1}}
\linespread{1.1}

\begin{document}
\title{Quantum Systems as Coalgebras}
\author{Luis S. Barbosa \and Liu Ai \and Renato Neves}
\maketitle

\section{Introduction}

\subsection{Quantum computing}

\nevComment{\underline{Goal of this subsection}: To convince the
  reader of the importance of quantum computing, and to give a first
  introduction to quantum systems.  \underline{Useful references}:
  \cite{nielsen2002quantum,NM08,ying16}. }

\subsection{Objectives}

The goals of the present work are,
\begin{enumerate}
\item to provide a unifying view of the
  current models for quantum transition systems;
\item and to formally relate them in terms of expressivity.
\end{enumerate}
\nevComment{We need to explain why these objectives are relevant.
  Note that subsequently we will use them to provide uniform, useful
  notions to quantum transition systems. Which notions for
  such systems are useful is still a matter of
  discussion.}

\subsection{What coalgebras bring into the game}

\nevComment{\underline{Goal of this subsection}: To tell the reader in
  which ways coalgebras can help better understand transition systems
  and develop tools for their analysis.  We will need to explain that
  coalgebras are a very useful framework for providing
  uniform views of (apparently) different types of transition
  systems. \underline{Useful references}:
  \cite{rutten2000,Jacobs16,sokolova,neves17}. }

\subsection{Document structure and notation}

\nevComment{\underline{Goal of this subsection}: To guide the reader
  along the document so that he can get a broad picture of our
  exposition at an early stage. Setting (non-standard) notation at a
  common point is also useful to the reader, since he then knows where
  to look when unfamiliar notation appears.}

\section{A Primer on Quantum Computing}

\subsection{What makes it different from classical paradigms}

\nevComment{\underline{Goal of this subsection}: To explain to the
  reader why quantum computing is a challenging thing, and why the
  problems that we are addressing are non-trivial. References from
  subsection 1.1 can also be used here.  Emphasise the problem of
  superposition and observation. We set the basic machinery of quantum
  computing (\emph{e.g.} Hilbert spaces, unitary matrices) in this subsection.}

\subsection{A general view of its current formalisms}

\nevComment{\underline{Goal of this subsection}: To give the reader a
  broad view of what is currently being done in regard to semantics of
  quantum systems. This includes not only transition systems, but also
  programming languages \cite{selinger04,hasuo17,ying16}, circuit
  formalisms \cite{nielsen2002quantum}, and process algebras
  \cite{jorrand04,ying09}. In the following sections our focus will be
  mainly on transition systems.}

\subsection{Examples of quantum systems}

\nevComment{\underline{Goal of this subsection}: To introduce the
  reader to a stock of examples of quantum systems. They will be used
  for illustrating different formalisms of quantum transition
  systems. \emph{N.b.}: in subsequent work we will also use them as
  guiding principles for developing new models of quantum systems.}

\section{A Primer on Coalgebra}

\subsection{Basic notions}

\nevComment{\underline{Goal of this subsection}: To introduce the reader
to Coalgebra. References from subsection 1.2 can be used here.}


\subsection{Coalgebraic analysis of probabilistic systems: an
  illustration of the abstraction power that is provided by coalgebras}

\nevComment{\underline{Goal of this subsection}: To convince the
  reader that Coalgebra is a powerful abstraction tool for studying
  and developing different types of state-based transition systems.
  This justifies our choice of the coalgebraic approach for
  accomplishing the goals presented above. Our illustration should concern
  probabilistic systems because they are close to quantum ones. The
  main reference for this subsection is \cite{sokolova}.}


\section{Current Formalisms for Quantum transition systems}

\subsection{Quantum automata}

\nevComment{\underline{Goal of this subsection}: To introduce the
  reader to quantum automata and related notions. The main reference for this is the
  survey \cite{hirvensalo11}.}

\subsection{Coalgebras over set-based state spaces}

\nevComment{\underline{Goal of this subsection}: To introduce the
reader to Ichiro's and Ogawa's formalism (which involves the quantum
branching monad) \cite{hasuo17,ogawa14}. The notions of behavioural
equivalence proposed in these two references should also be mentioned, as they are one
clear source of motivation to our goals.}

\subsection{Coalgebras over Hilbert-based state spaces}

\nevComment{\underline{Goal of this subsection}: To introduce the
  reader to Roumen's formalism \cite{F14}. The notions of minimisation
  that were proposed and the duality between Schrodinger and Heisenberg should
  also be mentioned; for the same reason than above.}


\section{Relation between formalisms of quantum transition systems}

\subsection{A uniform view of quantum transition systems}

\nevComment{\underline{Goal of this subsection}: To provide the reader with a uniform
view of quantum automata \cite{hirvensalo11} using a coalgebraic approach. Extend this
view to Ichiro's and Roumen's coalgebras if possible; and if not explain why.}

\subsection{A hierarchy of formalisms for quantum transition systems}

\nevComment{\underline{Goal of this subsection}: Establish a hierarchy
  of formalisms for quantum transition systems. In the coalgebraic
  approach this ammounts to having monomorphic natural transformations
  between the functors of interest (see \cite[Section 4.4]{sokolova}).
}

\section{Summary}

\nevComment{\underline{Goal of this subsection}: To give a broad
  picture of the results achieved in the previous section. In
  particular, a diagram depicting the hierarchy established, and the
  (dis)advantages of each formalism. At the end of this section, the
  reader should have,
  \begin{enumerate}
  \item a clear overview of existing formalisms for quantum transition systems,
  \item and how they relate to each other in terms of expressivity,
  \item a clear understanding of which formalism to choose for modelling a certain quantum
    system,
  \item and a coalgebraic definition that captures all different
    notions of quantum transition system.
  \end{enumerate}
  The reader should be convinced that the paper gave him enough tools to be able to study and develop
  useful notions for quantum transitions systems \underline{in a uniform manner}. 
}

\noindent
\nevComment{{\large \underline{Deadline}: 10th of October 2018.}}


\section{\dots}


%%Bibliography
\bibliographystyle{alpha}
\bibliography{the}

\end{document}