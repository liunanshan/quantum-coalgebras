\documentclass[a4paper, 11pt]{article}
\usepackage{comment} % enables the use of multi-line comments (\ifx \fi) 
\usepackage{lipsum} %This package just generates Lorem Ipsum filler text. 
\usepackage{fullpage} % changes the margin
\usepackage{libertine}
\usepackage{color}
\newcommand{\nevComment}[1]{\textcolor{red}{RN: #1}}
\linespread{1.1}

\begin{document}
\title{Coalgebras meet Quantum Computing}
\author{Luis S. Barbosa \and Liu Ai \and Renato Neves}
\maketitle

\section{Introduction}

\subsection{The importance of quantum computing}

\nevComment{\underline{Goal of this subsection}: To tell the reader
why it is important to study quantum computing. Also, to give a
first introduction to quantum systems.  \underline{Useful references}:
\cite{nielsen2002quantum,NM08,ying16}. }

\subsection{What coalgebras bring into the game}

\nevComment{\underline{Goal of this subsection}: To tell the reader
in which ways coalgebras can help better understand quantum systems and develop
tools for their analysis. \underline{Useful references}:
\cite{rutten2000,Jacobs16}. }

\subsection{Objectives}

The goals of the present work are,
\begin{enumerate}
\item to provide a unifying view of the
  current models for quantum transition systems;
\item to formally relate them in terms of expressivity.
\end{enumerate}
\nevComment{We will also need to tell why these objectives are relevant.}

\subsection{Document structure and notation}

\nevComment{\underline{Goal of this subsection}: To guide the reader
along the document, so that he can get a broad picture of our
exposition at an early stage. Setting (non-standard) notation at
a common point is also useful to the reader.}

\section{A Primer on Quantum Computing}

\subsection{What makes it different from classical paradigms}

\nevComment{\underline{Goal of this subsection}: To explain to the
reader why quantum computing is a challenging thing, and why the
problems that we are addressing are non-trivial. References from
subsection 1.1 can also be used here. }

\subsection{A general view of its current formalisms}

\nevComment{\underline{Goal of this subsection}: To give the reader a
  broad view of what is currently being done in regard to semantics of
  quantum systems. This includes not only transition systems, but also
  programming languages \cite{selinger04,hasuo17,ying16}, circuit
  formalisms \cite{nielsen2002quantum}, and process algebras
  \cite{jorrand04,ying09}, etc \dots. Later on we will focus just on transition
  systems}

\subsection{Examples of quantum systems}

\nevComment{\underline{Goal of this subsection}: To introduce the reader
to a stock of examples of quantum systems. The latter will be used
for illustrating different models of quantum transition systems.}

\section{A Primer on Coalgebra}

\subsection{Basic notions}

\nevComment{\underline{Goal of this subsection}: To introduce the reader
to coalgebra. References from subsection 1.2 can be used here.}


\subsection{Coalgebraic analysis of probabilistic systems: an
  illustration of the coalgebraic approach to the study of state-based
  transition systems}

\nevComment{\underline{Goal of this subsection}: To convince the
  reader that Coalgebra is a powerful abstraction tool for studying
  and developing different types of state-based transition systems in
  a uniform manner.  This justifies why we took the coalgebraic
  approach for accomplishing goals presented above. Our illustration
  concerns probabilistic systems because they are close to quantum
  ones. The main reference for this subsection is \cite{sokolova}.}


\section{Current Formalisms for Quantum transition systems}

\subsection{Quantum automata}

\nevComment{\underline{Goal of this subsection}: To introduce the
  reader to quantum automata and related notions. The main reference for this is the
  survey \cite{hirvensalo11}.}

\subsection{Quantum coalgebras in the category of sets}

\nevComment{\underline{Goal of this subsection}: To introduce the
reader to Ichiro's and Ogawa's formalism (which involves the quantum
branching monad) \cite{hasuo17,ogawa14}. The notions of behavioural
equivalence proposed in these two references should also be mentioned; as they are one
source of motivation to our goals.}

\subsection{Quantum coalgebras in the category of convex sets}

\nevComment{\underline{Goal of this subsection}: To introduce the
  reader to Roumen's formalism \cite{F14}. The notions of minimisation
  that were proposed and the duality between Schrodinger and Heisenberg should
  be mentioned; they are sources of motivation to our goals.}


\section{Relation between formalisms of quantum transition systems}

\subsection{A uniform view of quantum transition systems}

\nevComment{\underline{Goal of this subsection}: To provide the reader with a uniform
view of quantum automata \cite{hirvensalo11} using a coalgebraic approach. Extend this
view to Ichiro's and Roumen's coalgebras.}

\subsection{A hierarchy of formalisms for quantum transition systems}

\nevComment{\underline{Goal of this subsection}: Establish a hierarchy
  of formalisms for quantum transition systems. In the coalgebraic
  approach this ammounts to having monomorphic natural transformations
  between the functors of interest (see \cite[Section 4.4]{sokolova}).}

\section{Summary}

\nevComment{\underline{Goal of this subsection}: To give a broad
  picture of the results achieved in the previous section. In
  particular, a diagram depicting the hierarchy established, and the
  (dis)advantages of each formalism.}


\section{\dots}


%%Bibliography
\bibliographystyle{alpha}
\bibliography{the}

\end{document}